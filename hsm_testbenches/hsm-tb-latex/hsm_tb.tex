%
% Portuguese-BR vertion
% 
\documentclass{article}

\usepackage{hsm_tb}
% Use longtable if you want big tables to split over multiple pages.
% \usepackage{longtable}
\usepackage[utf8]{inputenc} 
\usepackage[spanish]{babel} % Uncomment for portuguese

\sloppy

\graphicspath{{./pictures/}} % Pictures dir
\makeindex
\begin{document}

\DocumentTitle{Documento de Pruebas de Proyecto}
\Project{Módulo hardware de criptografía ligera orientado al internet de las cosas}
\Organization{CIATEQ}
\Version{Versión 1.0a}

\capa
\newpage

%%%%%%%%%%%%%%%%%%%%%%%%%%%%%%%%%%%%%%%%%%%%%%%%%%
%% Revision History
%%%%%%%%%%%%%%%%%%%%%%%%%%%%%%%%%%%%%%%%%%%%%%%%%%
\section*{\center Histórico de Revisiones}
  \vspace*{1cm}
  \begin{table}[ht]
    \centering
    \begin{tabular}[pos]{|m{2cm} | m{7.2cm} | m{3.8cm}|} 
      \hline
      \cellcolor[gray]{0.9}
      \textbf{Date} & \cellcolor[gray]{0.9}\textbf{Descripción} & \cellcolor[gray]{0.9}\textbf{Autor(es)}\\ \hline
      \hline
      \small xx/xx/xxxx & \small <Descripción> & \small <Autor(es)> \\ \hline      
      \small xx/xx/xxxx &
      \begin{small}
        \begin{itemize}
          \item item;
          \item item;
        \end{itemize}
      \end{small} & \small <Autor(es)> \\ \hline 
    \end{tabular}
  \end{table}

\newpage

% TOC instantiation
\tableofcontents
\newpage

%%%%%%%%%%%%%%%%%%%%%%%%%%%%%%%%%%%%%%%%%%%%%%%%%%
%% Document main content
%%%%%%%%%%%%%%%%%%%%%%%%%%%%%%%%%%%%%%%%%%%%%%%%%%
\section{Introducción}

\subsection{Vista Geral del Documento}
En este documento se redacta la información necesaria para realizar las pruebas al módulo hardware para seguridad basado en el estándar~\cite{1059-1993-std:1994}.

  \begin{itemize}
   \item \textbf{Requisitos funcionales -} Lista de todos los requisitos funcionales.
   \item \textbf{Requisitos no funcionales -} Lista de todos los requisitos no Funcionales.
   \item \textbf{Dependencias -} Conjunto de dependencias de IP-cores previstos.
   \item \textbf{Notas -} Lista de notas presentadas en el documento.
   \item \textbf{Referencias -} Lista de todos los textos referenciados en el documento.
  \end{itemize}

  % inicio das definições do documento
  \subsection{Definiciones}
    \FloatBarrier
    \begin{table}[H]
      \begin{center}
        \begin{tabular}[pos]{|m{5cm} | m{9cm}|} 
          \hline
          \cellcolor[gray]{0.9}\textbf{Término} & \cellcolor[gray]{0.9}\textbf{Descripción} \\ \hline
          Requisitos Funcionales & Requisitos que hacen funcional al sistema, son las capacidades que debe tener el sistema entregado.  \\ \hline
					Requisitos Técnicos & Requisitos del sistema que definen características referentes a técnicas, algoritmos, tecnologías y especificidades de los requerimientos funcionales.  \\ \hline
          Requisitos No Funcionales & Requisitos de los módulos entregables.  Se refieren a las capacidades no funcionales del sistema como un todo y que especifican necesidades del usuario final.  \\ \hline
          Dependencias & Requisitos de reuso de IP-cores, describiendo las funciones que cada uno de estos módulos debe realizar. \\ \hline
        \end{tabular}
      \end{center}
    \end{table}  
  % fim

  % inicio da tabela de acronimos e abreviacoes do documento
  \subsection{Acrónimos y abreviaciones}
    \FloatBarrier
    \begin{table}[H]
      \begin{center}
        \begin{tabular}[pos]{|m{2cm} | m{12cm}|} 
          \hline
          \cellcolor[gray]{0.9}\textbf{Sigla} & \cellcolor[gray]{0.9}\textbf{Descripción} \\ \hline
          FR      & Requisito Funcional  \\ \hline
					TR      & Requisito Técnico  \\ \hline
          NFR     & Requisito No Funcional  \\ \hline
          D       & Dependencia  \\ \hline
        \end{tabular}
      \end{center}
    \end{table}  
  % fim

  % inicio da descriao de prioridades de requisitos
  \subsection{Prioridades de los Requisitos}
    \FloatBarrier
    \begin{table}[H]
      \begin{center}
        \begin{tabular}[pos]{|m{2cm} | m{12cm}|} 
          \hline
          \cellcolor[gray]{0.9}\textbf{Prioridad} & \cellcolor[gray]{0.9}\textbf{Característica} \\ \hline
          Importante     & Requisito para que el sistema sea entregado.  \\ \hline
          Esencial       & Requisito que debe ser implementado para que el sistema funcione.  \\ \hline
          Deseable       & Requisito que no compromete el funcionamento del sistema.  \\ \hline
        \end{tabular}
      \end{center}
    \end{table}  
  % fim

  % inicio dos requisitos Funcionales
  \section{Requisitos Funcionales}
	
En un sistema HSM, un controlador maestro envía peticiones de servicios continuamente al HSM, entonces, el HSM responde a dichas peticiones con servicios de seguridad. Debido a que hay muchas solicitudes del controlador maestro, el HSM debe responder a las solicitudes muy rápidamente. Para este propósito, el microcontrolador y otros módulos FPGA deben estar altamente optimizados~\cite{evita-hsm:2012}. 

En esta etapa del desarrollo del HSM, no se determina cómo tomará forma el progreso del software que utilizará los servicios del HSM, los requisitos existentes definen las funcionalidades del sistema y los algoritmos que se implementarán en FPGA para el cifrado, el \textit{hashing}, firma digital y la generación de llaves.

    \subsection{Requisitos Funcionales}
    \begin{functional}
     % format \requirement{name}{description}{priority}
     \requirement{fr1}
		  {Cada llave debe ser usada por una sola función criptográfica}
      {The HSM ensures that each cryptographic key is only used for a single cryptographic function and only for its intended purpose.}
      {Importante}
    
     \requirement{fr2}
		  {Cálculo de resumen (\textit{hash}) de mensajes}
      {permite la generación y verificación firmas digitales hash y adicionalmente HMAC.}
      {Importante}
			
     \requirement{fr3}
      {Cifrador asimétrico}
      {Descripción breve y objetiva.}
      {Importante}
			
     \requirement{fr4}
      {Cifrador simétrico}
      {Descripción breve y objetiva.}
      {Importante}
			
     \requirement{fr5}
      {Cálculo de números pseudo-aleatorios}
      {Descripción breve y objetiva.}
      {Importante}
			
			\requirement{fr6}
			{Proveer un contador monotónico}
			{Descripción breve y objetiva.}
			{Importante}
			
			\requirement{fr7}
			{Creación de llaves internamente}
			{Descripción breve y objetiva.}
			{Importante}
			
    \end{functional}

  \subsection{Requisitos Técnicos de los Requisitos Funcionales}
  
    \begin{technical}
      \techrequirement
			{Requisitos Técnicos de FR\ref{fr1}}
			{
      \begin{itemize}
        \item[$-$]{Los datos que serán utilizados llegan bloque a bloque al HSM}
        \item[$-$]{El algoritmo utilizado es SHA-256}
				\item[$-$]{El \textit{hash} se envía a una dirección especificada previamente}
      \end{itemize}
      }
			
      \techrequirement
			{Requisitos Técnicos de FR\ref{fr2}}
      {
      \begin{itemize}
        \item[$-$]{Los datos que serán utilizados llegan bloque a bloque al HSM}
        \item[$-$]{El algoritmo utilizado es SHA-256}
				\item[$-$]{El \textit{hash} se envía a una dirección especificada previamente}
      \end{itemize}
      }
			
			\techrequirement
			{Requisitos Técnicos de FR\ref{fr3}}
      {
      \begin{itemize}
        \item[$-$]{El algoritmo utilizado es AES-128}
      \end{itemize}
      }
			
			\techrequirement
			{Requisitos Técnicos de FR\ref{fr4}}
      {
      \begin{itemize}
        \item[$-$]{Los datos que serán utilizados llegan bloque a bloque al HSM}
        \item[$-$]{El algoritmo utilizado es SHA-256}
				\item[$-$]{El \textit{hash} se envía a una dirección especificada previamente}
      \end{itemize}
      }
			
			\techrequirement
			{Requisitos Técnicos de FR\ref{fr5}}
      {
      \begin{itemize}
        \item[$-$]{Los datos que serán utilizados llegan bloque a bloque al HSM}
        \item[$-$]{El algoritmo utilizado es SHA-256}
				\item[$-$]{El \textit{hash} se envía a una dirección especificada previamente}
      \end{itemize}
      }
			
			\techrequirement
			{Requisitos Técnicos de FR\ref{fr6}}
      {
      \begin{itemize}
        \item[$-$]{Los datos que serán utilizados llegan bloque a bloque al HSM}
        \item[$-$]{El algoritmo utilizado es SHA-256}
				\item[$-$]{El \textit{hash} se envía a una dirección especificada previamente}
      \end{itemize}
      }
			
			\techrequirement
			{Requisitos Técnicos de FR\ref{fr7}}
      {
      \begin{itemize}
        \item[$-$]{Los datos que serán utilizados llegan bloque a bloque al HSM}
        \item[$-$]{El algoritmo utilizado es SHA-256}
				\item[$-$]{El \textit{hash} se envía a una dirección especificada previamente}
      \end{itemize}
      }
    \end{technical}    
 
\section{Requisitos no Funcionales}
% Esta seção apresenta a lista de Requisitos No Funcionales do projeto.

  \begin{nonfunctional}
    \requirement{nfr1}
		{Nombre del Requisito}
    {Descripción breve y objetiva.}
    {Importante}

    \requirement{nfr2}
		{Nombre del Requisito}
    {Descripción breve y objetiva.}
    {Importante}
  \end{nonfunctional}

\section{Dependencias}
  % Esta seção apresenta uma lista dos IP-cores disponíveis % para reuso e que devem ser adotados no desenvolvimento % deste projeto.

  \begin{dependencies}
    \dependency{Nombre del IP-\textit{core}}
		{Descripción breve y objetiva del IP-\textit{core} y referencia al documento.}

    \dependency{Nombre del IP-\textit{core}}
		{Descripción breve y objetiva del IP-\textit{core} y referencia al documento.}
  \end{dependencies}  

% Optional bibliography section
% To use bibliograpy, first provide the ipprocess.bib file on the root folder.
\bibliographystyle{ieeetr}
\bibliography{bibliography}

\end{document}